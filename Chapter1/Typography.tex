\documentclass{article}

\usepackage{mathtools}  % useful for paired delimiters

\title{Latex I Typography}
\author{Kyle Hernandez}
\date{September 1, 2023}
% Paired delimiter
\DeclarePairedDelimiter{\floor}{\lfloor}{\rfloor}
\DeclarePairedDelimiter{\ceiling}{\lceil}{\rceil}


\begin{document}
\maketitle
\
The Pigeonhole Principle is the
k
= 1 case of the Extended Pigeonhole
Principle.


Set: A collection of distinct objects.
A set can be expressed as a set containing 2, 5, and 7, or as a set with one member, {2, 5, 7}, where the members are distinct from each other.


Member: An element belonging to a set.
The pigeonhole principle asserts that sets must be distinct from each other, as seen in the example of people and days of the week.

P = {Annie, Batul, Charlie, Deja, Evelyn, Fawwaz, Gregoire, Hoon
}
D = {Sunday, Monday, Tuesday, Wednesday, Thursday, Friday,
Saturday
}

Cardinality: The number of elements in set A.
The number of
members of a finite set X is called its cardinality or size

Mapping: A relationship between elements of two sets.
A function f :X -> Y is sometimes called a mapping fromX to Y, and f is said to map an element.

Equal: Having the same value or quantity.
Theorem 2.4 states that when two integers, m and n, are squared and one result is added, the resulting number is 2k + 1.

Not equal: Having different values or quantities.
 If we want to specify that A
50 essential discrete mathematics for computer science
is a subset of B, but is definitely not equal to B

Floor: The greatest integer less than or equal to x.
If x is any real number, we write x for the greatest integer less than or equal to x.

Ceiling: The smallest integer greater than or equal to x.

Fraction: A number that represents a part of a whole.
A fraction with the same value but a smaller numerator and
a smaller denominator. In 2.7, the square root of 2 is irrational.

Sequence: An ordered list of elements.
A sequence of terms can be denoted by a repeated variable with different
numerical subscripts, algebraic expresions.
\end{document}
