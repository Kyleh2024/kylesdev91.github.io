\documentclass{article}

\usepackage{mathtools}  % useful for paired delimiters

\title{Unit 2: Basic Proof Techniques Homework}
\author{Kyle Hernandez}
\date{September 11, 2023}
% Paired delimiter
\DeclarePairedDelimiter{\floor}{\lfloor}{\rfloor}
\DeclarePairedDelimiter{\ceiling}{\lceil}{\rceil}


\begin{document}
\maketitle
\
2.1  An odd integer is any integer that can be written as 2k + 1, where k is also an integer. However  -1 is not an odd integer due to that there is no value in the 2n + 1



2.3  The product of two odd numbers, a and b, is an odd number. The integer expression can be expanded using the distributive property 2(2mn + m + n) + 1 =  4mn + 2m + 2n + 1 


2.5  A number is irrational if it is not rational; that is, if it cannot be expressed as the ratio of any two integers or any fraction in two integers. 3 square root 2 is irrational 


2.7 Imagine taking a rectangular box, like a typical dice, and stretching it in one of its dimensions, say, vertically. This stretching will result in a rectangular prism that is longer in one dimension, but still has six faces.

Now, let's take this stretched rectangular prism and taper the top and bottom faces, creating a shape that resembles a hexagonal prism. This shape has six rectangular faces and two hexagonal faces.

Finally, we can take this hexagonal prism and further taper the ends, making the hexagonal faces smaller until they become regular polygons with seven sides each. This transformation results in a polyhedron with seven faces, resembling a die but with one elongated dimension.

2.9 (A) To prove that if c and d are perfect squares, then cd is a perfect square, we can use the property that the product of two perfect squares is also a perfect square.

Let c = a squared and d = b squared, where a and b are integers.

Then cd = a squared times b squared  = a times b squared, which is a perfect square.

Therefore, if c and d are perfect squares, then cd is also a perfect square.




2.9 (B) To disprove the statement that if cd is a perfect square and c not equal to d, then c and d are perfect squares, we can provide a counterexample.

Let c = 4 and d = 9. Both c and d are not perfect squares.

However, cd = (4)(9) = 36, which is a perfect square.

Therefore, the statement is false.


2.9 (C)  To prove that if c and d are perfect squares such that c > d, and x squared = c and y squared = d, then x greater than y, we can use the fact that the square root of a perfect square is always greater than the square root of any smaller perfect square.

Let c = a squared and d = b squared, where a and b are integers and c > d.

Since c greater than  d, we have a squared greater than b squared.

Taking square roots of both sides, we get absolute value a  > absolute value b.

Since a and b are integers, we can conclude that a > b or a < -b.

Since x = a and y = b, we have x > y.

Therefore, if c and d are perfect squares such that c > d, and x squared = c and y squared 2 = d, then x > y.
2.11 The "proof" is incorrect because it assumes that if the product of two numbers is greater than zero, then both numbers must be positive. However, this is not true, as one number could be negative while the other is positive. Therefore, the conclusion that x greater than -y is not necessarily valid.

2.13 (a) For every positive real number x, there exist distinct real numbers a and b such that a squared = x and b squared = x.

2.13 (b) For every positive even number x, there exist prime numbers a and b such that x = a + b.

2.15 Pigeonhole Principle. This principle states that if we have n+1 objects to distribute into n containers, then at least one container must contain more than one object. In this context, we can think of the 6 people as the "objects" and the two categories of knowing and not knowing each other as the "containers." Since each person can either know or not know the others, we have two containers. If the container of knowing contains more than two people, then X knows at least 3 people. On the other hand, if the container of not knowing contains more than two people, then X does not know at least 3 people.

\end{document}
